\begin{problem}{E: Cellular Automata}

Cellular automata is a fairly common interest among the lonely computer scientist.
A special type of rulesets is of central interest because of the famous variant, Conway's Game of Life (GoL).
These rulesets consist of two kinds of rules, survival rules and birth rules.
Both rules consist of a number corresponding to the count of live cells adjacenet to the current cell of interest.
Adjacent means left, right, up, down, and diagonals (of distance one).
A survival rule means that if a cell is alive and the count of adjacent cells matches the number for any of the given survival rules, then the cell continues to live.
A birth rule means that if a cell is dead and the count of adjacent cells matches the number for any of the given birth rules, then the cell becomes alive.
The rules of GoL can be described succinctly as S23/B3, survive if there are 2 or 3 live cells, birth if there is 2 live cells.

Although GoL is the most popular of these variants, there are several just as interesting rulesets.
One such pattern is S23/B36 called HighLife.
Although these rules are very interesting they do not encapsulate all of cellular automata.
Given two states of a cellular automaton your job is determine if a ruleset of the S/B form can describe the evolution from the first state to the second state.

It is possible to have more then one ruleset that describes the evolution.
If this does occur you are to break ties by first favoring the ruleset with the minimum number of characters required to describe the ruleset and second by favoring the ruleset with the minimum numbers describing each rule.
\end{problem}

\begin{formalin}
The input starts with an integer $k$ ($1 < k < 100$) the number of cases.
The next line has one integer $s_i$ ($1 < s_i < 100, 0 <= i < k$) representing the side lengths of the grid.
The following $s_i$ lines have $s_i$ integers defining the first state.
There will be one extra line for readability of the sample input, and then next $s_i$ lines have $s_i$ integers defining the second state.
A cell of the grid is considered alive if its value is 1, it is considered dead if its value is 0.
This pattern repeats for $k-1$ additional times (for $k$ total cases).
Cells outside of the given grid are considered dead.
\end{formalin}

\begin{formalout}
For each test case, output the ruleset describing the state transition from the first state to the second state of the form S*/B*. If no such ruleset is possible then output "Not Possible."
\end{formalout}

\begin{datain}
3
5
0 0 0 0 0
0 0 1 1 0
0 0 1 0 1
0 0 1 0 0
0 0 0 0 0

0 0 0 0 0
0 0 1 1 0
0 1 1 0 0
0 0 0 1 0
0 0 0 0 0
5
0 0 0 0 0
0 1 1 1 0
0 1 0 1 0
0 1 1 1 0
0 0 0 0 0

0 1 0 1 0
1 0 0 0 1
0 0 0 0 0
1 0 0 0 1
0 1 0 1 0
5
0 0 0 0 0
0 0 1 1 0
0 0 1 0 1
0 0 1 0 0
0 0 0 0 0

1 0 0 0 0
0 0 1 1 0
0 0 1 0 0
0 0 0 1 0
0 1 0 0 0
\end{datain}

\begin{dataout}
S23/B3
S/B2
Not Possible.
\end{dataout}

