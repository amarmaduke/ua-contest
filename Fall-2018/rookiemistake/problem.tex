\begin{problem}{A: Rookie Mistake}
In a game of chess, the rook piece is able to move in a single straight line up, down, left, or right but not diagonally.
The rook can move in this direction as many places as it likes until it is either:
1. Blocked by another piece on his team then the rook must stop in the square on the board before colliding with the teammate.
2. Hits the edge of the chess board then the rook must stop on the last piece.
3. Collides with a piece on the enemy team then the rook will kill the enemy player, and take its place on the board.

It apears that after you thought you won the game, your rook forgot to kill the enemy's king.
Since the king is not dead, you know one of your rooks is to blame.
This means you know you have at least one rook on the board but in a game of chess, you can have up to two total rooks.
Figure out where your rook is and remind him to finish the objective.
If that rook is unable to kill the king find your second rook and make sure the job is completed.
Make sure you do not kill your own king.

The position of pieces are determined from the X and Y positions of the board where
top-left=(0,0)     top-right=(7,0)     bottom-right=(7,7)

Note to players that uderstand the real game of chess: These situations may not be valid game ending moves in the real game of chess but these are just fun litle problems.
\end{problem}

\begin{formalin}
The input consists of 8 lines containing 8 space separated letters.
The letters are:
R: A rook is occupying this position
K: Your king is occupying this position
$: The enemy king is occupying this position
N: There are no pieces occupying this position
\end{formalin}

\begin{formalout}
On a single line that is space separated,

Output the X position of the rook you wish to move (zero indexed)
Output the Y position of the rook you wish to move (zero indexed)
Output the direction (U, D, L, R) you wish to move the rook in (Up, Down, Left, and Right respectively)
Output the number of spaces the rook must move to kill the king.
\end{formalout}

\begin{datain}
N N N N N N N N
N N N K N N N N
N N N N N N N N
N N N N N N N N
N N N N N N N N
N N N N N N N N
N R N N N N $ N
N N N N N N N N
\end{datain}
\begin{dataout}
1 6 R 5
\end{dataout}

\begin{datain}
N N N N N N N N
N N N N N N N N
N N N N N N N N
N R N K N $ N N
N N N N N N N N
N N N N N N N N
N N N N N N K N
N N N N N R N N
\end{datain}
\begin{dataout}
5 7 U 4
\end{dataout}
