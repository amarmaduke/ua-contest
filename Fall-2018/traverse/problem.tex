\begin{problem}{A: Che-xploration-ss}
Albert does not really know much about chess and would like to explore the game to learn more.
As a challenge, Albert's friend decided to give him a board with obstacles and a single piece.
Albert was tasked to find the minimum number of moves needed to reach any other cell.
However, Albert negotiated with his friend that if the minimum number of moves would be larger than $9$, then he would report the number as \texttt{*} instead.

Of course, Albert does not know how any of the pieces move!
His friend decided that he should only learn about the knight, the bishop, the rook, and the queen to start.
Recall that the knight is allowed to move two spaces horizontally or vertically and then one space in the opposite horizontal or vertical direction, forming an \textit{L} shape.
The rook is allowed to move vertically or horizontally as much as it wants while staying on the board.
The bishop is allowed to move diagonally (both left and right) as much as it wants while staying on the board.
The queen can move as either a rook or a bishop.
An obstacle on the board will be represented by an \texttt{X} and will prevent movement.
You can think of this obstacle as an immovable pawn under your own control (i.e. you can not capture it).

Unfortunately, neither Albert or his friend are very good at making sure Albert's answers are correct.
That is where you come in.
Albert's friend has asked you to write a program that will produce the correct answer so that Albert can easily check his attempt.
\end{problem}

\begin{formalin}
The first line of input consists of one character that is either \texttt{B}, \texttt{R}, \texttt{K}, or \texttt{Q} representing if the piece is a bishop, rook, knight, or queen respectively followed by two integers, $u$ and $v$ ($0 \le u, v < 8$), the starting position of the piece.
The following 8 lines of input consist of 8 space separated characters.
These characters are either \texttt{.} or \texttt{X} representing an open space and an obstacle respectively.
\end{formalin}

\begin{formalout}
Output 8 lines with 8 space separated characters each that represents the input grid with labels for the minimum number of moves to reach each location.
\end{formalout}

\begin{datain}
Q 5 4
. . . . . . . .
. . . . . . . .
. . . . . . . .
. . . . . . . .
. . . . . . . .
. . . . . . . .
. . . . . . . .
. . . . . . . .
\end{datain}
\begin{dataout}
2 2 2 2 1 2 2 2
1 2 2 2 1 2 2 2
2 1 2 2 1 2 2 1
2 2 1 2 1 2 1 2
2 2 2 1 1 1 2 2
1 1 1 1 0 1 1 1
2 2 2 1 1 1 2 2
2 2 1 2 1 2 1 2
\end{dataout}

\begin{datain}
R 0 0
. X . . . . X .
. . . X X . X .
X X X . . . X .
. X . . X X . .
. X . X . . . X
. X . . . X . .
. . X X X . X .
. . . X . . . .
\end{datain}
\begin{dataout}
0 X 3 4 4 4 X *
1 2 2 X X 5 X *
X X X 6 6 5 X *
. X 8 7 X X * *
. X 9 X * * * X
. X 9 * * X * *
. . X X X * X *
. . . X * * * *
\end{dataout}
