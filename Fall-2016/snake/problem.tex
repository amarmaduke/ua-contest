\begin{problem}{C. Snake}
Snake loves card games - especially ones that involve betting. Perhaps that has to do with how often he wins. He's a sneaky snake, capable of tipping the odds in his favor. One of his favorite games is poker. Poker is a card game played with a standard 52-card deck. The deck consists of unique cards that belong to four suits. The suits are spades, clubs, hearts, and diamonds. Each suit has thirteen unique card ranks. In order of increasing value, the ranks are 2 through 10 (inclusive) and four "face" cards - jack, queen, king, and ace. Each card in the deck is unique, so it is guaranteed no two players ever end up with the same card. In poker the only exception to the aforementioned ordering is that the ace may also be used as a rank lower than the king depending on the other cards in a player's hand.

The game of poker is played with multiple players having hands of five cards. In most variants there are rounds of betting, but we'll disregard those rounds for now. The value of a hand corresponds to the best 'category' it belongs to. The categories and descriptions are as follows in order from the best hand to the worst:

\begin{tabular}{|l|l|} 
\hline
 Royal flush & Ace King Queen Jack 10 all of the same suit. \\
  & No tiebreaker. \\ 
 \hline
 Straight flush & Five cards of sequential rank all of the same suit. \\
  & Tie goes to the hand with the highest card.* \\ 
 \hline
 Four-of-a-kind & Four cards in the hand have the same rank. \\
  & Tie goes to the hand whose four cards are higher. \\ 
 \hline
 Full house & Three cards of the same rank and two other cards of the same rank. \\
  & Tie goes to the hand whose three cards are higher. \\ 
 \hline
 Flush & Five cards of the same suit. \\
  & Tie goes to the hand with the highest card. \\ 
 \hline
 Straight & Five cards of sequential ranks. \\ 
  & Tie goes to the hand with the highest card.* \\ 
 \hline
 Three-of-a-kind & Three cards in the hand have the same rank. \\ 
  & Tie goes to the hand whose three cards are higher. \\
 \hline
 Two pair & Two cards of the same rank and two other cards of the same rank. \\ 
  & Tie goes to the hand who has a pair with higher cards than the other hand. \\
  & If neither hand wins from that comparison, consider the 'kicker' card (described below). \\
 \hline
 Pair & Two cards of the same rank. \\ 
  & Tie goes to the hand whose pair has a higher rank. If this results in a tie, \\
  & consider the three 'kickers' (described below) \\
 \hline
 
 \multicolumn{2}{l}{*If the ace is used as the low rank card in a straight it will have the lowest value in a tiebreaker.} \\
\end{tabular}

 For pairs and two-pairs, the 'kicker' cards are the ones not involved in the pairs. If the kickers must be used as a tiebreaker, start by comparing the kickers with the highest ranks. If the ranks are the same, move on to the next kicker. Do this until one of the kickers has a higher rank or until there are no more kickers left. The winner is the hand whose kicker was higher, or the hand is a tie if all kickers had the same rank.
 
 Note that a five-high straight (ace, 2, 3, 4, 5) will not beat any other straight since the ace is considered to have a rank lower than the 2.
 
 Snake is playing poker with another person. As mentioned before, Snake is sneaky. He is able to sneak a peek at the other player's hand, and substitute one of his cards with any card still available in the deck (those not in his or his opponent's hand). Given his opponent's hand and his hand, what is Snake's best possible outcome for the hand?

\end{problem}

\begin{formalin}
The input consists of two lines that each contain a hand of five cards. 
The first hand belongs to the snake's opponent, and the second to the snake.
A card consists of two tokens. The first is a string which denotes the card's rank. 
For cards 2 - 10 the input contains the number of the card. For non-numerical cards, 
jack is input as J, queen as Q, king as K, and ace as A. The next token is a character 
which denotes the suit of the card. Spades are denoted by S, clubs by C, hearts by H, 
and diamonds by D. All non-numerical input will be in upper-case characters.
\end{formalin}

\begin{formalout}
Output the best possible outcome for the snake after replacing any one of his cards with 
any other available card (excluding the others in his hand and his opponent's hand). 

If he is able to win, output ``Yes.''
If he's only able to tie the game, output ``Tie.''
If he can only lose, output ``Lose.''
\end{formalout}

\begin{datain}
K H Q H J H 10 H 9 H
A S K S Q S J S 9 S
\end{datain}
\begin{dataout}
Win.
\end{dataout}

\begin{datain}
2 H 3 C 4 D 5 S 6 D
A S 2 C 3 H 5 C 9 D
\end{datain}
\begin{dataout}
Lose.
\end{dataout}

\begin{datain}
A H A D Q S Q H K D
A S A C Q D Q C 9 D
\end{datain}
\begin{dataout}
Tie.
\end{dataout}