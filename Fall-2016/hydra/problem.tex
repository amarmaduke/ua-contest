\begin{problem}{Hydra}
\end{problem}
In 1982 Laurence Kirby and Jeff Paris discovered an incredibly powerful species of hydra.
Hydras are known for their ability to regrow heads that are forcefully removed, and this monster is no different.
Susan has become fascinated with species of hydra after reading about Kirby and Paris' discoveries.

She knows a bit more about the state of things than you and is well aware that you will be hopefully defeated by a fully grown hydra or even a baby hydra of Kirby and Paris' original variation.
Nonetheless she is interested in a weaker species of hydra and specifically in the minimum number of cuts it would take to slay it.
She's smart enough, again, to only come to you with a theoretical question of slaying baby hydras of this weakened variation.

First though, we need to know how hydra's regrow their heads.
Consider a hydra to be made up of it's core, joint-pints, and heads.
You can only cut off a head, not a joint-point, or the core.
When you do chop off a head, the hydra will sense it not from the immediately connecting joint-point, but \textit{that} joint-point's joint-point.
To make that more clear, if you imagine a chain with four links, the first link is the core, the second and third links are joint-points, and the final link the head.
When you remove the final link, the second link senses it (not the third).
The second link will then send a neuron through the joint-point that alerted it of its head being chopped off.
It will find all links from that third reporting joint-point and grow a complete copy of it out of itself.
Then, all stubbed joint-points will become heads.
The core of the hydra can never become a head and the hydra is considered dead when only the core remains.

Here are some diagrams to get the point across:

\begin{verbatim}
   Core - Joint - Joint - Head(cut)

   Core - Joint - Head
            |
           Head
\end{verbatim}
\begin{verbatim}
   Core - Joint - Joint - Head
                    |
                   Head(cut)

   Core - Joint - Joint - Head
            |
          Joint - Head
\end{verbatim}
\begin{verbatim}
   Core - Joint - Head
            |
          Joint - Head(cut)

           Head
            |
   Core - Joint - Head
            |
           Head
\end{verbatim}

Simple!
Susan is really counting on you to give her the answer for the minimum number of cuts to slay these baby hydras.
Don't let her down!

\begin{formalin}
The first line of input will be two integers, $n$ ($2 \leq n \leq 6$), the total number of heads, joints, and the core combined.
The following $n-1$ lines of input will be the links between the parts of the hydra.
Each of these lines will be two integers.
The integer $1$ will always represent the core.

The maximum depth of the hydra (the number of joints between a head and the core) will be no more than two.
\end{formalin}

\begin{formalout}
Output one integer, the minimum number of cuts to reduce a hydra to nothing but its core.
\end{formalout}

\begin{datain}
4
1 2
2 3
3 4
\end{datain}
\begin{dataout}
8
\end{dataout}

\begin{datain}
5
1 2
2 3
2 4
4 5
\end{datain}
\begin{dataout}
16
\end{dataout}
