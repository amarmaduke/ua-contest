\begin{problem}{D. Phoenix}
The phoenix is the most beautiful and majestic of mythical beasts.
As I am sure you are well aware, a phoenix is gifted with the ability to be reborn from its own ashes.
However, few know of the intricacies of this ability.

Indeed, a phoenix has a total lifespan shared across all of its many rebirths.
Each time a phoenix rebirths itself it can pick a particular color for its feathers.
Different species of phoenix have different colors.
Each color has a certain innate beauty to other phoenix but also an innate penalty to the total lifespan.
After each rebirth a phoenix might also be revitalized and have their lifespan increased by some constant amount.

A legendary phoenix watcher named Bernard has asked you to assist him in his general observations.
He has studied phoenix for a very long time and now wants to know just how beautiful a phoenix can in its lifespan.
He is going to give you the computed lifespan, increased vitality on rebirth, beauty of colors, and penalty of colors.
You are to tell him the maximum beauty the phoenix can achieve in its lifespan.
\end{problem}

\begin{formalin}
The first line of input will be one integer, $n$ ($1 \leq n \leq 100$), the number of colors.
The second line of input will be two integers, $h$ ($1 \leq h \leq 1000$), the starting lifespan of the phoenix, and, $v$ ($0 \leq v < h$), the increased vitality per rebirth.
The next line will consist of $n$ integers $b_i$ ($0 \leq b_i \leq 1000$) the associated beauty score for the color $i$.
The final line will consist of $n$ integers $p_i$ ($0 \leq p_i < h$) the associated penalty for the color $i$.
\end{formalin}

\begin{formalout}
Output the maximum possible beauty for the phoenix in its lifespan.
If this value is unbounded output ``Infinite.''
\end{formalout}

\begin{datain}
3
10 1
2 3 9
2 3 9
\end{datain}
\begin{dataout}
27
\end{dataout}

\begin{datain}
2
10 10
1 1
1 1
\end{datain}
\begin{dataout}
Infinite.
\end{dataout}
