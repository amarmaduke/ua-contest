\begin{center}
\large
\textbf{The 6th Annual University of Akron Programming Competition}

\vspace{.25in}

presented by

The University of Akron Computer Science Department

Association for Computing Machinery Student Chapter

\vspace{.25in}

April 9, 2015
\end{center}

\textbf{Rules:}
\begin{enumerate} \itemsep10pt \parskip0pt \parsep0pt
\item There are six questions to be completed in four hours.
\item C, C++, and Java are the only languages available.
\item Data is read from Standard Input and output is sent to Standard Output. Do not prompt for input values in the code you submit to be judged. Do not attempt to read from or write to any files.
\item All programs are submitted as source code only. Submitting compiled information (binary, .class) will result in a Compilation Error penalty.
\item Non-standard libraries cannot be used in your solutions. The Standard Template Library (STL) and C++ string libraries are allowed. The standard Java API is available, except for those packages that are deemed dangerous by contest officials (e.g., that might generate a security violation).
\item The input to all problems will consist of multiple test cases. The Sample Input listed on the problem description is not intended to be a comprehensive input set. The input is guaranteed to adhere to the descriptions in each problem; you do not need to check for invalid input.
\item The output to all problems should conform to the format shown in the Sample Output for that problem, including but not limited to capitalization, whitespace, etc.
\item Use of personal electronics during the competition is cause for disqualification. Cell phones must be turned all the way off (not silent mode, not airplane mode, etc.). You may use any written notes, books, or reference materials you bring with you.
\item Programming style is not considered in this contest. You can code in any style or with any level of documentation your team prefers.
\item All communication with the judges will be handled through PC2. All judges’ decisions are final.
\end{enumerate}
