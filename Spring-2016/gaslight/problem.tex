\begin{problem}{E: Gas Light}

Andrew is driving to Akron for the weekly Friday ACM programming practice when suddenly his gas light comes on.
He's not sure whether he has enough gas to make it to practice or not.
Given the amount of miles he can drive before he runs out of gas he needs help finding out if he can make it to Akron or not.

Each city has one or more roads to other cities.
Each road between cities is exactly 1 mile long.
None of the roads are one-way.
It is guaranteed to be possible to reach any city from any other city through some series of roads.

Given that Andrew starts on city 1, and there are $n$ cities with the $n$th city representing Akron, can Andrew make it to Akron before he runs out of gas?
\end{problem}

\begin{formalin}
The first line of input will contain 3 integers, $n$ ($1 \leq n \leq 100$), the number of cities, $k$ ($n - 1 \leq k \leq \frac {n(n-1)}{2}$), the number of roads, and $m$ ($1 \leq m \leq 20$), the number of miles until Andrew runs out of gas.
The next $m$ lines contain the descriptions of the roads.
The $i$th road is described by 2 integers, $a_i$ and $b_i$ ($1 \leq a_i, b_i \leq n$), the two cities connected by the road.
There will be no duplicated roads.
\end{formalin}

\begin{formalout}
Output ``Yes'' if Andrew can make it to Akron or ``No'' if he cannot, followed by a new line.
\end{formalout}

\begin{datain}
5 4 4
1 2
2 3
3 4
4 5
\end{datain}
\begin{dataout}
Yes
\end{dataout}

\begin{datain}
5 4 3
1 2
2 3
3 4
4 5
\end{datain}
\begin{dataout}
No
\end{dataout}
