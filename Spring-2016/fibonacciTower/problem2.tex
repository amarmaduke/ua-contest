\begin{problem}{C: Fibonacci Tower}

Helga has always had a fancination with the fibonacci sequence. The fibonacci sequence begins with two defined numbers: 0 & 1. After these first two values, each number in the sequence is defined as the sum of the two numbers that preceed it. This can be described as $n_{i}$ = $n_{i-1}$ + $n_{i-2}$. The first six numbers in the sequence are 0, 1, 1, 2, 3, 5.

Helga, however, has grown board with this sequence and has devised a way to combine three fibonacci-like sequences that relly on one another for the next value in their sequnce.

These three sequences, $A$, $B$, and $C$ are formally defined as:
$A_{i}$ = $B_{i-1}$ + $C_{i-2}$
$B_{i}$ = $C_{i-1}$ + $A_{i-2}$
$C_{i}$ = $A_{i-1}$ + $B_{i-2}$

The starting values of these sequences will be:
$A_{0}$ = 0, $A_{1}$ = 1
$B_{0}$ = 1, $B_{1}$ = 1
$C_{0}$ = 1, $C_{1}$ = 2

Desipte her double checking, Helga is unsure if she is correctly calculating the numbers in sequences $A$, $B$, and $C$. Given these defined rules, and a number n, find the $n$th number in each of these sequences. 
\end{problem}

\begin{formalin}
Input consists of a single integer value ($1 \leq n \leq 40$), the number of each sequence to print.
\end{formalin}

\begin{formalout}
Output the space separated $n$th number in sequences $A$, $B$, and $C$.
\end{formalout}

\begin{datain}
3
\end{datain}
\begin{dataout}
2 2 2
\end{dataout}

\begin{datain}
7
\end{datain}
\begin{dataout}
16 13 13
\end{dataout}

\begin{datain}
1
\end{datain}
\begin{dataout}
0 1 1
\end{dataout}