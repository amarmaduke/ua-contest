\begin{problem}{B: Fibonacci}
Eric really likes his recurrence relations.
One of his favorites is the Fibonacci recurrence relation.
That is, $F_n = F_{n-1} + F_{n-2}, F_0 = 0, F_1 = 1$.
Of course, the initial conditions for Fibonacci can take one more than just those values.
In fact, the Fibonacci initial conditions are regularly stated as either $(0, 1)$, $(1, 1)$ or $(1, 2)$.
You can imagine that initial conditions that maintain the Fibonacci sequence are going to faithfully generate the sequence.

But this got Eric thinking.
What if we wrote the three separate recurrence relations down with the three more common initial conditions:

$a_n = a_{n-1} + a_{n-2}, a_0 = 0, a_1 = 1$,

$b_n = b_{n-1} + b_{n-2}, b_0 = 1, b_1 = 1$,

$c_n = c_{n-1} + c_{n-2}, c_0 = 1, c_1 = 2$.

Then, mix up the recurrences to make them all interdependent like this:

$a_n = b_{n-1} + c_{n-2}, a_0 = 0, a_1 = 1$,

$b_n = c_{n-1} + a_{n-2}, b_0 = 1, b_1 = 1$,

$c_n = a_{n-1} + b_{n-2}, c_0 = 1, c_1 = 2$.

While maintaining the same initial conditions.
This got Eric really wondering, what would be the $n$th triple ($a_n$, $b_n$, $c_n$) of this interdependent relation?
Well, Eric happens to be a busy guy, so he doesn't have time to figure it out on his own, so he's hired you to do the job.

\end{problem}

\begin{formalin}
The input consists of one integer $n$ ($1 \leq n \leq 40$).
\end{formalin}

\begin{formalout}
Output the resulting triple, $a_n$ $b_n$ $c_n$.
\end{formalout}

\begin{datain}
6
\end{datain}
\begin{dataout}
8 10 8
\end{dataout}
