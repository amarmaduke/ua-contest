\begin{problem}{B: Word Hops}

Tina has a curious computer that operates on strings.
Her computer is able to move from one space separated word to another instantly.
It can also move from one character to another instantly.
Tina tells you that she considers a ``move'' an instance of the computer moving it's address pointer either by one character or by one word.
She also tells you that the computer always starts pointing at the very first character which is the zero address of the programs memory.

Although Tina has managed to get her hands on some interesting hardware, she's not really a software person.
She's asked you to help her out in writing a program to determine the minimum number of moves required to get to an address.
That is, if the computer has the state ``Hello World'', with the target address of 7, then it could move the pointer once from the 0 address to the 6 address (now pointing at W), and then once more to the 7 address.
That means it would take the computer two moves to get the 7 address with the initial state ``Hello World''.

Given the number of words, the target address, and the initial state of the computer, your job is to figure out the minimum number of moves for Tina.

\end{problem}

\begin{formalin}
The input consists of two lines.
The first line has two integers, $w$ ($1 \leq w \leq 100$), the number of words and $t$ ($1 \leq t \leq 1000$), the target index.
The second line is a sequence $w$ space separated words with no special characters.
\end{formalin}

\begin{formalout}
Output the minimum number of moves required to reach the target index and the character at the target index.
\end{formalout}

\begin{datain}
2 6
Hello World
\end{datain}
\begin{dataout}
1 W
\end{dataout}


\begin{datain}
6 20
This is the best sentence ever
\end{datain}
\begin{dataout}
7 t
\end{dataout}
