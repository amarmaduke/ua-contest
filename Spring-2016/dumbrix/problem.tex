\begin{problem}{D: Shrek and Dumbrix}
Shrek loves puzzle games.
One of his favorite games is Numbrix.
In this game, you are given a 9 by 9 starting grid with some positions containing values and some blank.
The point of the game is to fill in the blank spots so that there is a continuous chain of numbers between 1 and 81.
In other words, all values from 2 to 80 are directly connected to the values 1 greater and 1 less than themselves.
Connections are only allowed through single movements on the vertical and horizontal axes (i.e. no diagonal movement).
It is guaranteed that each Numbrix grid has only 1 possible solution.

Shrek likes Numbrix, but he thinks it might be too hard for you.
He has decided to relax the rules a little bit so you might have a chance of solving the puzzles too.
In his revised version of the game, Dumbrix, he provides two additional guarantees.
The given board always contains the numbers 1 and 81.
Also, for any continuous sequence of values between 1 and 81 of length $\geq 6$ at least one value in the sequence exists in the given board.
\end{problem}

\begin{formalin}
The input consists of 9 lines each containing 9 integer values.
The value at each line $i$ and position $j$, $v_{i,j}$ ($0 \leq v_{i,j} \leq 81$), represents its corresponding board position's value.
Positions with a value of 0 represent unknown values.
\end{formalin}

\begin{formalout}
Output the solved board with a single space between each value in a row and a new line after each row.
\end{formalout}

\begin{datain}
1 0 19 0 37 0 55 0 73
0 17 0 35 0 53 0 71 0
3 0 21 0 39 0 57 0 75
0 15 0 33 0 51 0 69 0
5 0 23 0 41 0 59 0 77
0 13 0 31 0 49 0 67 0
7 0 25 0 43 0 61 0 79
0 11 0 29 0 47 0 65 0
9 0 27 0 45 0 63 0 81
\end{datain}
\begin{dataout}
1 18 19 36 37 54 55 72 73
2 17 20 35 38 53 56 71 74
3 16 21 34 39 52 57 70 75
4 15 22 33 40 51 58 69 76
5 14 23 32 41 50 59 68 77
6 13 24 31 42 49 60 67 78
7 12 25 30 43 48 61 66 79
8 11 26 29 44 47 62 65 80
9 10 27 28 45 46 63 64 81
\end{dataout}

\begin{datain}
35 0 0 0 41 0 0 0 47
0 0 0 0 0 0 0 0 0
0 0 29 0 57 0 51 0 0
0 0 0 27 0 0 0 0 0
7 0 9 0 0 0 63 0 67
0 0 0 0 0 0 0 0 0
0 0 13 0 21 0 73 0 0
0 0 0 0 0 0 0 79 0
1 0 0 0 17 0 0 0 81
\end{datain}
\begin{dataout}
35 36 39 40 41 42 43 46 47
34 37 38 55 54 53 44 45 48
33 30 29 56 57 52 51 50 49
32 31 28 27 58 61 62 65 66
7 8 9 26 59 60 63 64 67
6 11 10 25 24 23 72 71 68
5 12 13 20 21 22 73 70 69
4 3 14 19 18 75 74 79 80
1 2 15 16 17 76 77 78 81
\end{dataout}
