\begin{problem}{C: Bowling Scoring}

If you’ve ever been confused about how bowling is actually scored then this is the problem for you! Bowling consists of 10 frames and in each frame you have up to two chances to knock down ten pins. If you knock them all down, you get what is known as a strike (symbolized by a capital X) and move on to the next frame. If you finish knocking down all pins on your second attempt in a frame, you get a spare (symbolized by a forward slash). All other results are represented with integer values from 0 to 9.

Strikes and spares are what add difficulty in calculating the score. When you get a strike, the values for the next two attempts are counted twice. If you get a spare, the next attempt is counted twice. The only case where this can be difficult is in the 10th frame. If you get a strike in the final frame, you will always have two additional attempts to improve your score, giving you the opportunity to knock down two full sets of pins. Getting a spare in the final frame would allow one attempt to knock down as many pins as possible.
\end{problem}

\begin{formalin}
The input begins with an integer $n$ indicating the number of cases. Each case is a single string that describes the results of playing all 10 frames of bowling. This string will only contain either an X, a /, or integers 0 to 9. 
\end{formalin}

\begin{formalout}
For each case, output the final score.
\end{formalout}

\begin{datain}
3
XXXXXXXXXXXX
43127190112750900090
4312719011275090009/X
\end{datain}

\begin{dataout}
300
61
72
\end{dataout}
