\begin{problem}{B: River Delta}

Rivers tend to be called the lifeblood for the humanity on earth.
Typically we congregate around rivers and consider them as the principal source of water.
There is a special name for certain parts of certain rivers when they meet with the ocean called a river delta.
A river delta occurs when the river sediments soil and other materials at its mouth to the ocean.

One could guess that the longer a river exists the larger its delta will be.
Because deltas can be correlated to an expansion in viable living land for humans its of interest to scientists to determine the size of the river delta with respect to time.
Scientists have described a simple model as a start for this exploration and have asked your help in its design and implementation.

To simplify things we think of a river delta as an equilateral triangle.
We define the rate of growth for the area of the delta per hour as r.
We assume that there is no delta initially and after one hour the area of the delta is r.
The scientists need you to develop a model based on this description for a given r and predict the height of the triangle representing the delta after twenty-four hours.
\end{problem}

\begin{formalin}
The input starts with an integer $k$ ($1 < k < 100$) the number of cases.
The next $k$ integers $r_i$ ($0 < r_i \leq 50, 0 \leq i < k$) correspond to the ith rate of growth.
\end{formalin}

\begin{formalout}
For each test case, output the area of the delta to four decimal digits of precision.
Always include four digits after the decimal place, even if they are not significant.
\end{formalout}

\begin{datain}
2
2
5
\end{datain}

\begin{dataout}
9.1180
14.4169
\end{dataout}

