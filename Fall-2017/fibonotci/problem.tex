\begin{problem}{C: Recurring Recurrences}
Lily discovered an interesting fact about her favorite recurrence relation, the Fibonacci sequence.
She discovered that the solution to the sequence can be written as $f_n = a_1 \alpha^n + a_2 \beta^n$ for some constants $a_1$, $a_2$, $\alpha$, and $\beta$.
She's already worked out the math to figure out what those constants are.
Lily had an epiphany about this particular form for the solution.
That is, it must be the case that there is a $\mathcal{O}(\log n)$ solution for finding the $n$th Fibonacci number.

Unfortunately, the equation Lily found above involves real arithmetic so round off error quickly prevents her from computing dramatically large numbers.
She wants \textit{really} large Fibonacci numbers.
Lily understands that this is impractical even for 64-bit integers so she's content calculating these numbers mod $7001$.
She doesn't want to stop there, though, she's confident that if a fast solution for the Fibonacci recurrence relation exists then one must exists for \textit{all} recurrence relations of a similar type.
That is, given a recurrence relation of the form $f_n = c_1 f_{n-1} + c_2 f_{n-2}$, with $f_2 = a$ and $f_1 = b$ she wants a quick solution for very large values of $n$.
Note that in order to prevent real arithmetic from rearing its ugly head again she is only interested in recurrence relations where $c_1$, $c_2$, $a$, and $b$ are all positive integers.

Lily is stuck on just how to go about achieving this goal.
She's come to you for help because you're a super genius.
Help Lily figure out the $n$th number mod $7001$ in a collection of recurrence relations.
\end{problem}

\begin{formalin}
The first line of input is one integer, $q$ ($1 \leq q \leq 1000$), the number of queries.
The next $q$ lines consists of five integers, $n$ ($3 \leq n \leq 10^8$), $c_1$, $c_2$, $a$, and $b$. ($1 \leq c_1, c_2, a, b, \leq 7000$).
The integer $n$ is the $n$th number of the recurrence relation that you must compute.
The integers $c_1$ and $c_2$ are the coefficients in the recurrence relation.
The integers $a$ and $b$ are the starting values for the recurrence relation.
\end{formalin}

\begin{formalout}
Output one integer for each query, the $n$th number in the recurrence relation, modulus $7001$.
\end{formalout}

\begin{datain}
4
6 1 1 1 1
700 1 1 1 1
3 1 2 3 4
10 10 10 1 1
\end{datain}

\begin{dataout}
8
3766
11
3092
\end{dataout}
