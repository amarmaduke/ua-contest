\begin{problem}{E: The Height is Right}
Trees are very important data structures in computing. Their shapes and structures may differ a lot, but they all have certain things in common. Each tree begins with a root node that may have child nodes connected to it. The child nodes may have their own child nodes and are considered to be subtrees of the overall tree. Each non-root node of the tree has exactly one parent node that it is connected to. Nodes that do not have children are called leaf nodes.

We define the height of a tree as the longest of all paths from the root node to the leaf nodes. The root node is at level 0, its children are at the level 1, their children are at level 2, etc.

Some trees have additional rules about child nodes. Many people are familiar with binary trees, which are trees whose nodes have at most 2 child nodes. Oftentimes this is referred to as having a branching factor or degree of 2.

\addtolength{\tabcolsep}{10pt}
\begin{center}
\begin{tabular}{c c}\\
7 node binary tree height 3 & 7 node binary tree height 2\\
 \Tree [.A [.B [.D [.F ] [.G ] ] [.E ] ].B [.C ] ].A & \Tree [.A [.B [.D ] [.E ] ].B [.C [.F ] [.G ] ] ].A \\
\end{tabular}
\end{center}
\addtolength{\tabcolsep}{-10pt}

As observable in the trees above, a tree with the same nodes as another tree may be arranged in a way such that some leaf nodes are much further from the root than others.

Given a branching factor and a number of nodes, find the minimum height of any valid tree constructed with those constraints.

\end{problem}

\begin{formalin}
The input consists of one line containing two integers.
The first, $b$ ($1 \leq b \leq 10^9$), is the balancing factor of the tree.
The second, $n$ ($1 \leq n \leq 10^9$), is the amount of nodes in the tree.
\end{formalin}

\begin{formalout}
On a single line, output the minimum height of the tree.
\end{formalout}

\begin{datain}
2 7
\end{datain}
\begin{dataout}
2
\end{dataout}

\begin{datain}
2 8
\end{datain}
\begin{dataout}
3
\end{dataout}