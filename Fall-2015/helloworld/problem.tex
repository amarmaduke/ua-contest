\begin{problem}{A: Hello World!}

``Hello World!'' is ubiquitous in today's programming world.
What introductory work doesn't start with a good ``Hello World!'' problem?
Heck, even some competitive programming contests start with ``Hello World!'' problems!
There are a few things that are essential to make a problem like this truly introductory.
It has to be simple to understand, introduce the basics, and give the user something nice to look at.
A bit of comedy on the side doesn't hurt either.

Let's revisit the famous ``Hello World!'' with a little bit of a twist, or I guess in our case, a shift.
Print out the famous string, but shifted by the input value!

\end{problem}

\begin{formalin}
The input consists of a stream of integers $n$ ($1 \leq n \leq 100$).
Each integer is an independent case, the input data ends when $n = 0$.
\end{formalin}

\begin{formalout}
For each input case output the string ``Hello World!'' sans quotes, shifted by the input value.
\end{formalout}

\begin{datain}
2
0
\end{datain}

\begin{dataout}
llo World!He
\end{dataout}

