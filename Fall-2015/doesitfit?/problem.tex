\begin{problem}{B: Does it Fit?}
Johnny recently had to move, and if, you've ever moved before, you can understand that there are lots of boxes involved. Now that Johnny is done moving he wants to condense all the boxes so that they take up less room. The problem is, there are a lot of boxes to combine. Johnny knows how to program a little bit and starts working on developing a program that will help him. For this problem, he only wants to know if two boxes can be combined in some way. This means that either box could nest inside the other one if some rotation of the boxes causes all the dimensions of the inner box to be smaller than the corresponding dimension of the outer box.
\end{problem}

\begin{formalin}
The input starts with an integer indicating the total number of cases. Each case consists of six integers on a single line. The first three integers represent the dimensions of the first box, and the last three integers represent the dimensions of the second box.
\end{formalin}

\begin{formalout}
If it is possible to orient the boxes such that one box fits within another box, print "YES", otherwise, print "NO". Output for each test case should be on its own line.
\end{formalout}

\begin{datain}
2
1 2 3 4 5 6
1 1 1 2 2 2
\end{datain}

\begin{dataout}
YES
NO
\end{dataout}

