\begin{problem}{D: Shots Fired}
Jessie and James are professional sharpshooters that are best known for being able to make bullets collide in mid-air.
As you might expect, this requires a lot of trial and error when practicing.
In an effort to make practices more productive, they have hired you to write a program that determines if the bullets will collide.
\end{problem}

\begin{formalin}
The first line of input contains a single integer that specifies the number of test cases.
Each test case is on its own line and consists of twelve integers.
These integers represent the initial positions and the vectors at which the bullets will be fired.
The first three integers, $x$ $y$ $z$ ($-1000000000 \leq x, y, z \leq 1000000000$), are the starting coordinates of the bullet and the second three integers, $u$ $v$ $w$ ($-10 \leq u, v, w \leq 10$), represent the vector describing the direction and velocity of the bullet.
The second set of six integers represent the second bullet in the same fashion as the first.
It is guaranteed that the two bullets will not have the same starting position.
\end{formalin}

\begin{formalout}
For each case, if the bullets will collide output ``HIT'', otherwise output ``MISS''.
The bullets can only collide on integer coordinates and we only consider integer time steps for both bullets' movement. 
\end{formalout}

\begin{datain}
2
2 4 6 1 1 1 1 3 5 2 2 2
1 1 1 -1 -2 -3 1 2 3 1 1 1
\end{datain}

\begin{dataout}
HIT
MISS
\end{dataout}

