\begin{problem}{A: Test Memorization}
Janet needs to memorize a bunch of different words for her test.
The professor informed her that every word would be its own question.
Obviously, Janet decided that she wanted to memorize as many cards as possible to maximize her score.
Unfortunately, there are a \textit{lot} of cards that she needs to memorize so figuring out which ones to study first by hand is very cumbersome.

Each card has one word on it that starts with an ASCII lowercase alpha character (\texttt{'a'-'z'}) and continues with either an ASCII lowercase alpha character or a numeral (\texttt{'0'-'9'}).
Janet was able to write a small program herself to label the cards with how many seconds she thinks each one will take to memorize.
She also knows how many seconds she has available to study before the test.
Help Janet out by telling her which cards she should study in order to maximize her score on the test!
\end{problem}

\begin{formalin}
The first line of input consists of two integers, $n$ ($1 \leq n \leq 10^4$), the number of cards, and $k$ ($1 \leq k \leq 10^{10}$), the total number of seconds available to study. 
The next $n$ lines consist of a string $s_i$, the name of the $i$th card, and an integer, $t_i$ ($1 \leq t_i \leq k$), the time in seconds needed to memorize the $i$th card.
\end{formalin}

\begin{formalout}
Output one integer, the maximum number of cards that can be memorized in the time available to study.
\end{formalout}

\begin{datain}
5 10
rememberful 8
joykingly 4
hep 2
mayhaps 2
belive 5
\end{datain}
\begin{dataout}
3
\end{dataout}

\begin{datain}
4 10
heppily 8
mutin 2
terseful 9
googlex 8
\end{datain}
\begin{dataout}
2
\end{dataout}