\begin{problem}{F: Parents Loving Sequence}
Johnny loves playing games with lists of numbers.
One of his favorite games is to sum the numbers up in the list.
Johnny is a bit of a prodigy.
Gauss himself would be proud.
His mother, Abigail, decided that Johnny needs to push his talents farther so that he can grow intellectually.

First Abigail decided that Johnny should start summing two lists at a time instead of just one.
This seemed at first like a decent enough difficulty but Johnny quickly overcame it.
Abigail decided that it was too time consuming to write out a whole list each time Johnny wanted a new problem.
So, she modified the problem slightly to give Johnny two formulas for generating the lists instead of the lists themselves.
The first formula would consist of $n$ coefficients and $n$ starting numbers.
The second formula would consist of $m$ coefficients and $m$ starting numbers.
The next number in the first list, the $n+1$st number, would be the linear combination of the coefficients and the preceding numbers.
For example, given starting numbers $a_i$ and coefficients $c_i$:

$a_0 = 1$, $a_1 = 2$, $a_3 = 3$, $c_0 = 0$, $c_1 = 1$, $c_2 = 2$

then, $c_3 = c_2a_2 + c_1a_1 + c_0a_0 = 2\times 1 + 1\times 2 + 0\times 3 = 4$.
$c_4$ would then be calculated with $c_3$, $c_2$, and $c_1$ (and the corresponding $a_i$s).
Moreover, Abigail would give Johnny a number of terms $t_n$ for the first list and $t_m$ for the second list that Johnny should compute.
Thus, after computing the two lists up to $t_n$ and $t_m$ elements respectively, he would then sum the lists.

Much to Abigail's chagrin and amazement Johnny quickly conquered this task as well.
It seemed that she would need to increase the difficulty with something less to do with numbers and more to do with manipulating the lists.
Abigail's next trick was to have Johnny sum up the longest shared list between the two lists instead of the lists individually.

She explained to Johnny what she meant by shared in the following way:
Johnny was allowed to remove an element from a list, but he had to find the fewest number of deletions such that the two lists where equal.
Johnny then had to sum just one of the lists, since they would have the same length and the same numbers.

Abigail was proud of herself and her son.
He didn't take to this problem as easily as the first, but all the same he managed it after a few months.
Resolute, Abigail decided one final modification, but she's requested your help in making sure that she gets the right answers in order to check Johnny's work.

Abigail wants to ask Johnny to sum a range of the list instead of the entire list.
Johnny should be able to answer $q$ queries by range of the list, not just a few.
You're job is to write a program that will give the answer to Abigail for each query so that she can double check Johnny's work.
\end{problem}
\newpage
\begin{formalin}
The input begins with two integers $n$ ($1 \leq n \leq 100$), the number of coefficients, and $t_n$ ($1 \leq t_n \leq 1000$), the total number of terms in the first list.
The next line is $2n$ integers, the first $n$ are the coefficients, $c_i$ ($-10 \leq c_i \leq 10$), in order from first to last.
The second $n$ integers are the starting values for the list, $a_i$ ($-10 \leq a_i \leq 10$), in order from first to last.
The next line is another two integers $m$ ($1 \leq m \leq 100$), the number of coefficients, and $t_m$ ($1 \leq t_m \leq 1000$), the total number of terms in the second list.
The next line is $2m$ integers, shaped similarly to the previous list of coefficients and starting values.
The next line is one integer, $q$ ($1 \leq q \leq 10000$), the number of queries.
The following $q$ lines are two integers, $l_i$ ($l_i > 0$) and $r_i$ ($r_i > 0$), which form a range $[l_i, r_i]$ (specifically, $l_i < r_i$) that Johnny should sum over.
If $l_i$ or $r_i$ are larger than the length of the shared list then they should be rounded down to match only the largest valid range for the shared list.
If the shared list is empty then the query range should be ignored and the result should be $0$ for each query.
\end{formalin}

\begin{formalout}
For each query output one integer that is the sum from $l_i$ to $r_i$ of the shared list.
\end{formalout}

\begin{datain}
2 5
1 1 0 1
3 6
1 1 1 0 1 0
3
2 3
1 10
2 4
\end{datain}
\begin{dataout}
2
7
4
\end{dataout}

\begin{datain}
3 10
0 0 1 1 0 1
1 10
2 1
2
1 1
1 10
\end{datain}
\begin{dataout}
1
1
\end{dataout}
