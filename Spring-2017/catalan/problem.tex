\begin{problem}{B: The Parenren Language}

Searle has come up with a new interesting programming language.
It consists of only paranthesis characters, $($ and $)$.
The left most paranthesis, $($, will put the number $1$ on a virtual stack.
Having two left most parenthesis next together, like $(($, will still put the number $1$ on a virtual stack and perform a nand operation.
Having three left most parentehsis next together, like $((($, will still put the number $1$ on the virtual stack and perform a nor operation.
The program evaluator is greedy, so the nor operation consisting of three left paranthesis will be performed always over the other two if possible.

A nand operation will pop off two numbers from the stack, ``and'' them, and then ``not'' the result.
So for example, 1 nand 1 will give 0, and 0 nand 1 will give 1.
A nor operation will pop off two numbers from the stack, ``or'' them, and then ``not'' the result.

After each nand or nor operation the result is put on the virtual stack.

A right most paranthesis will pop a number from the top of the virtual stack and print it to the screen.
On caveat is that every left paranthesis must be paired with a right parenthesis to make parsing the language simple.
Here are some example programs:

$((()()))$

$()()()$

$(()((()))())$

Searle is confident that this new programming language is turing complete, but he is curious about how expressive it is.
Given a number $n$, the number of paired paranthesis, Searle wants you to determine how many possible programs there are.
\end{problem}

\begin{formalin}
The input consists of one integer $n$ ($1 \leq n \leq 2000$).
\end{formalin}

\begin{formalout}
Output one integer, the number of possible programs, mod $10^7 + 14$.
\end{formalout}

\begin{datain}
4
\end{datain}
\begin{dataout}
14
\end{dataout}

\begin{datain}
1
\end{datain}
\begin{dataout}
1
\end{dataout}
