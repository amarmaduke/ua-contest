\begin{problem}{D: Photons for Sale}
Edward and his colleagues have been trying to transfer messages along photoluminescent wires.
It's a brand new technology and just recently they've been able to encode bits with photons.
A chip can hold a certain bit representation but there is a cost of moving the photons from one chip along the wire to a different chip.
The cost of moving from a chip with the bit representation 101 to another with 100 is a single unit of energy.
For every bit in the representation that changes between the initial chip and the target chip it costs one unit of energy.

Moving photons along the wire is particularly expensive but Edward only cares about getting to a certain bit representation from a starting chip.
Your job is to figure out the minimum cost it would take to get to all other possible chips from a given chip.
Not all chips are necessarily directly connected by wires and some chips might not be reachable at all.
\end{problem}

\begin{formalin}
The first line of input consists of two integers, $n$ ($1 \leq n \leq 100$), the number of chips, and $m$ ($1 \leq m < \frac{n(n-1)}{2}$), the number of wires connecting chips.
The next $n$ lines consist of a bit representation, a sequence of $0$s and $1$s, these sequences wont be longer than $100$ characters.
The next $m$ lines consist of two integers, the wire connection between two chips.
The final line of input is the initial chip.
\end{formalin}

\begin{formalout}
Output $n$ lines, the chips bit representation in order that they were given in the input, and the cost to reach that chip from the initial chip.
If a chip can not be reached from the initial chip then output $-1$ as the cost instead.
\end{formalout}

\begin{datain}
5 8
1000
1010
1111
0000
1011
0 1
0 2
1 2
1 3
1 4
2 3
2 4
3 4
0
\end{datain}
\begin{dataout}
1000 0
1010 1
1111 3
0000 3
1011 2
\end{dataout}
\begin{datain}
5 5
0000000000
0000000001
0000000010
0000000100
1111111111
0 1
0 3
1 2
2 3
3 4
0
\end{datain}
\begin{dataout}
0000000000 0
0000000001 1
0000000010 3
0000000100 1
1111111111 10
\end{dataout}
