\begin{problem}{A: Important Slime Research}
Tanya is working in a lab researching slime specimens. She found that every second every slime will split in two and that they can do so indefinitely.

This means that if at the first second there were 4 slimes, in one additional second there would be 8, and after two more seconds there would be 32.

Help Tanya find the number of slimes there will be after a specified number of seconds have passed, assuming that she always starts with 1 slime. 
\end{problem}

\begin{formalin}
Input consists of one integer, $n$ ($1 \leq n \leq 30$), the number of seconds Tanya waits before recording the new number of slimes.
\end{formalin}

\begin{formalout}
Output the number of slimes present after $n$ seconds.
\end{formalout}

\begin{datain}
1
\end{datain}
\begin{dataout}
2
\end{dataout}

\begin{datain}
2
\end{datain}
\begin{dataout}
4
\end{dataout}

\begin{datain}
12
\end{datain}
\begin{dataout}
4096
\end{dataout}

